Project \char`\"{}\+Bayqaw\char`\"{}

Installation ~\newline
You need to install E\+S\+P32 library from git\+: \href{https://github.com/espressif/arduino-esp32/blob/master/docs/arduino-ide/mac.md}{\tt https\+://github.\+com/espressif/arduino-\/esp32/blob/master/docs/arduino-\/ide/mac.\+md}

Then you need to install usb to U\+A\+RT driver ~\newline
\href{https://www.silabs.com/products/development-tools/software/usb-to-uart-bridge-vcp-drivers}{\tt https\+://www.\+silabs.\+com/products/development-\/tools/software/usb-\/to-\/uart-\/bridge-\/vcp-\/drivers}

\subsection*{B\+L\+E-\/\+Mesh network}

B\+L\+E-\/\+Mesh network is the special form of connection type where all elements of the network are nodes of the connection. Normally in the specification of the B\+L\+E-\/\+Mesh network is written that the devices that works in the mesh network have to be dual connection devices. By the other words, they got to have two B\+LE modules. In our case, we need to decrease cost of the project to the global minimum. As a result, we are introducing new protocol called \char`\"{}\+B\+Pro\char`\"{} that was developed to satisfy limitations and purposes of the Technical Task of this research and development project. \subsubsection*{Protocol \char`\"{}\+B\+Pro\char`\"{}}

\paragraph*{Overall}

B\+Pro protocol has several differences from standard B\+L\+E-\/\+Mesh network protocol\+:
\begin{DoxyItemize}
\item It uses only one B\+LE module to communicate.
\item It is slower than standard protocol.
\item It can handle more devices than standard protocol. ~\newline
 These characteristics of the B\+L\+E-\/\+Mesh network are archived by next architectural differences\+:
\item B\+Pro protocol uses each device as B\+L\+E-\/server and B\+L\+E-\/client to IO data.
\item B\+Pro protocol uses special identification algorithm based on unique service U\+U\+ID.
\item Because of asynchronous nature and strict U\+U\+ID based identification it can handle huge number of devices (theoretically 16$^\wedge$8 devices).
\end{DoxyItemize}

\paragraph*{Identification}

The identification number of the device is stored in the service U\+U\+ID of each module. Normally B\+LE device has service U\+U\+ID as next\+: 
\begin{DoxyCode}
4fafc201-1fb5-459e-8fcc-c5c9c331914b
\end{DoxyCode}
 These groups of bytes are used to encapsulate data as in our case. But we use it to identify priority of the devices.
\begin{DoxyEnumerate}
\item Highest priority\+: 1\+X\+X\+X\+X\+X\+XX -\/ Device directly connected to the Wi-\/\+Fi.
\item Lowest priority\+: X\+X\+X\+X\+X\+X\+X1 -\/ Device connected to the 7th device counting from the device with highest priority.
\end{DoxyEnumerate}

For example, there is pseudocode of the priority generation\+: 
\begin{DoxyCode}
is\_wifi\_connected = try\_to\_connect(ESSID, passwd)
if is\_wifi\_connected == TRUE:
  UUID = "1".join(["%s" % randint(0, 9) for num in range(0, 7)])
else:
  highes\_priority\_neighbour = scan\_for\_other\_devices(find\_the\_highest\_priority\_of\_neighbours)
  UUID = create\_UUID\_with\_less\_priority(highes\_priority\_neighbour)
return UUID
\end{DoxyCode}


By simple words\+:
\begin{DoxyEnumerate}
\item If device is connected to Wi-\/\+Fi -\/$>$ it has highes priority
\item If device is not connected -\/$>$ it searches neighbour with highes priority
\item Then it creates U\+U\+ID with priority 1 byte less than the priority of the highest priority neighbour
\item It means that the priority level is the distance of device to the Wi-\/\+Fi hotspot
\end{DoxyEnumerate}

\paragraph*{Algorithm}

The next list is the steps of the connection protocol of the B\+L\+E-\/\+Mesh network of the \char`\"{}\+Bayqaw\char`\"{} project.
\begin{DoxyEnumerate}
\item Device is turned on.
\item It starts to scan for wi-\/fi -\/$>$ if Wi-\/\+Fi is reachable go to point 5.
\item It scans neighbours with highes priority.
\item Device connects to neighbour with highest priority (parent neighbour) and creates new U\+U\+ID.
\item Device does job (reads sensor values or whatever) and takes information from daughter devices -\/ 2 minutes.
\item Device turns to client mode and starts to scan parent neighbour.
\item Device sends all data to parent neighbour, which is in server mode OR to tcp server if there is Wi-\/\+Fi.
\item Device goes to 5. 
\end{DoxyEnumerate}